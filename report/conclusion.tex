\section{Conclusion}

\paragraph{}
Our goal was to detect topics from 200 years of articles from \emph{la Gazette de Lausanne} and \emph{le Journal de Genève}. Our main effort has been to study, adapt and implement the algorithm describe in \cite{kdd05-ttm}. However, since this algorithm was designed for a smaller dataset, we had to, first, use a parallel Spark implementation and, then, try to find alternatives when the algorithm failed due to accuracy or complexity. This lead us to further inquire side topics like Information Theory, Probabilities and Hidden Markov models and, of course, improved our understanding of the Spark paradigm.

\paragraph{}
We formalised what topics should be and then built our algorithm and its results accordingly. The results can be of three types : 
\begin{itemize}
\item Extracted topics : finding topics over small periods of time.
\item Topics transitions : building transitions between topics over disparate periods of time.
\item Topics life-cycle : measuring the strength of a topic over a long period of time.
\end{itemize}
We compared the output of our algorithm with the main events of the last two centuries and were very fulfilled by the quality and precision of the results for all three categories. Furthermore, we successfully created easy to read visualisations of the output data.

\paragraph{}
This project was particularly useful for us to grasp the challenges and advantages of a parallel implementation and learn how to deal with huge amount of data. Collaborating on such a sizeable task was quite ambitious but we will all withdraw from this project improved and happy.